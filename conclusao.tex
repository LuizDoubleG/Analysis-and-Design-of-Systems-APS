% Prof. Dr. Ausberto S. Castro Vera
% UENF - CCT - LCMAT - Curso de Ci\^{e}ncia da Computa\c{c}\~{a}o
% Campos, RJ,  2022
% Disciplina: An\'{a}lise e Projeto de Sistemas
% Aluno: Luiz Miguel Guedes Gomes
 

\chapterimage{conclusoes.png} % Table of contents heading image
\chapter{Considera\c{c}\~{o}es Finais}


Os problemas enfrentados neste trabalho incluem a necessidade de desenvolver um trabalho resumido sobre um sistema grande e complexo, desta forma alguns aspectos e detalhes, por exemplo em relação aos requisitos, são deixados de lado. Além disso, por se tratar de um sistema fictício onde não se pode verdadeiramente conversar com os stakeholders, ocorre certa imprecisão nas entrevistas e na determinação de certos requisitos, inclusive acarreta na falta de determinação de restrições entre outros aspectos. Mas isso não impede que estes aspectos sejam estudados posteriormente e incluídos no projeto, tornando-o mais completo e ajudando a desenvolver ainda mais conhecimento sobre essa ampla área de conhecimento que a área de Analise e Projeto de Sistemas.

   \begin{figure}[H]
    \begin{center}
        \includegraphics[width=12cm]{twg_green_it_thumb.png}
        \caption{Sistema de gerenciamento e tecnologização do comércio de flores e plantas: Pollinator} \label{sistema}
    \end{center}
   \end{figure} 